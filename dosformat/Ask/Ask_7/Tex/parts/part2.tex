\section{}
Στο μέρος αυτό υλοποιούμε τις συναρτήσεις F0 = (AB+BC+CD+DE)' , F1= ABCD+E ,
F2 = F0+F1.  Χρησιμοποιώντας τις bitwise λογικές πράξεις της C, έχουμε τα
εξής: Για την F0 θέλουμε οποιαδήποτε 2 συνεχόμενα bits της θύρας εισόδου να
είναι στο λογικό 1.  Για τον σκοπό αυτό κάνουμε την πράξη and με μάσκα binary
11 = dec 3, και την οποιά κάνουμε shift προς τα δεξιά κατά μία θέση (c>>=1)
μέσα στο loop όσο το c είναι θετικό. Αν βρούμε δύο ίσα επιστρέφουμε 0. Σε άλλη
περίπτωση επιστρέφουμε 1.  Για την F1 θέλουμε είτε όλα τα 4 πρώτα bits να
είναι 1 είτε να έχουμε 0 το bit E. To bit E θα είναι 0 αν ο αριθμός μας είναι
μικρότερος ή ίσος του 15.  Ακόμη για να έχουμε τα ABCD όλα 1 θα πρέπει να
έχουμε μάσκα binary 11111 = dec 31.  Τέλος το F2 είναι αληθές είτε  F1 είτε
F2. Και περνάμε το αποτέλεσμα στην έξοδο  στο σωστό bit όπως ζητάει η εκώνηση
με διαδοχικές ολισθίσεις.

\noindent Κυρίως κώδικας:
\inputminted[linenos,obeytabs,fontsize=\footnotesize]{c}{files/part2.c}
