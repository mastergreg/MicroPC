\documentclass[a4paper,10pt]{article} \usepackage{anysize}
\marginsize{2cm}{2cm}{1cm}{1cm}
%\textwidth 6.0in \textheight = 664pt
\usepackage{xltxtra}
\usepackage{xunicode} \usepackage{graphicx}
\usepackage{color} \usepackage{xgreek} \usepackage{fancyvrb}
\usepackage{minted}
\usepackage{listings}
\usepackage{enumitem} \usepackage{framed} \usepackage{relsize}
\usepackage{float} \setmainfont[Mapping=TeX-text]{DejaVu Sans}
\begin{document}

\begin{titlepage}
\begin{center}
\begin{figure}[h] 
     \includegraphics[width=0.2\textwidth]{title/ntua_logo}
\end{figure}
\vspace{1cm}
\begin{LARGE}\textbf{ΕΘΝΙΚΟ ΜΕΤΣΟΒΙΟ ΠΟΛΥΤΕΧΝΕΙΟ\\[1.5cm]}\end{LARGE}
\begin{Large}
ΣΧΟΛΗ ΗΜ\&ΜΥ\\
Εργαστήριο Μικροϋπολογιστών\\[2cm]
3\textsuperscript{η} Εργαστηριακή Άσκηση\\[0.5cm]
\LARGE{Γενικό Θέμα}\\[0.5cm]
\large{Ακ. έτος 2011-2012}\\
\end{Large}
\vfill
\begin{flushright}
\Large \textit{Ομάδα C07:}\\[1cm]
\begin{tabular}{l r}
{Ελένη \textsc{Ευαγγελάτου}}&
{Α.Μ.: 03108050}\\
{Γρηγόρης \textsc{Λύρας}}&
{Α.Μ.: 03109687}\\
{Βασιλεία \textsc{Φραγκιαδάκη}}&
{Α.Μ.: 03108026}\\
\end{tabular}
\end{flushright}

\large\today\\
\end{center}
\end{titlepage}




\section*{} \setcounter{section}{1}
\subsection*{Άσκηση 1(ii)}\setcounter{subsection}{1}
Σ' αυτή την άσκηση κατασκευάζουμε ένα χρονόμετρο δευτερολέπτων που μετράει
από το 0 έως το 15 και στη συνέχεια ξαναρχίζει απ' την αρχή. Γι' αυτό το λόγο
έχουμε στον καταχωρητή Α το μετρητή, που λόγω της αρνητικής λογικής των leds
αρχικοποιούμε στο FFH (Hex) και στη συνέχεια αφαιρούμε 1, μέχρι το F0H
(δεκαδικό 15). Στην αρχή καλούμε τη ρουτίνα beep ενώ για την χρονοκαθυστέρηση
του ενός δευτερολέπτου φορτώνουμε στο διπλό καταχωρητή B-C το 1000Dec (03Ε8Η),
που η ρουτίνα delb πολλαπλασιάζει επί 1 msec, άρα συνολική καθυστέρηση 1 sec.
Οι ρουτίνες αυτές επηρρεάζουν τους καταχωρητές του συστήματος· συγκεκριμένα η
ρουτίνα delb εσωτερικά αποθηκεύει στη στοίβα όλους τους καταχωρητές και τον
καταχωρητή σημαιών, επομένως δε χρειάζεται να τους αποθηκεύσουμε εμείς πριν
την καλέσουμε ενώ για τη ρουτίνα beep δε χρειάζεται να τους αποθηκεύσουμε
δεδομένου ότι στην αρχή του προγράμματος τους αρχικοποιούμε κάθε φορά.
\inputminted[linenos,obeytabs,fontsize=\footnotesize]{oldasm}{../askhsh_1_ii.8085}
\subsection*{Άσκηση 2(i)}
Εδώ ζητείται να ανάβουμε και να σβήνουμε τα leds ανάλογα με την τιμή των
τεσσάρων  αριστερότερων και δεξιότερων διακοπτών της πόρτας 2000Η αντίστοιχα,
με καθυστέρηση $200$ έως $1700 msec$ και βήμα $100 msec$. Γι' αυτό το λόγο, αφού σε
κάθε συνδιασμό 4 διακοπτών έχουμε 16 δυνατές εισόδους με βήμα $+100msec$ κάθε
φορά, χρησιμοποιήσαμε τη συνάρτηση: $100 * x + 200$, όπου $x$ o δεκαδικός αριθμός
που διαβάζουμε από τα τέσσερα MSB και LSB των διακοπτών αντίστοιχα. Έτσι,
διαβάζουμε την είσοδο και αφού τα χωρίσουμε σε MSB και LSB, στη συνέχεια
πολλαπλασιάζουμε το διπλό καταχωρητή B-C επί 100 με κατάλληλες διαδοχικές
ολισθήσεις και προσθέσεις $(100= 2^2 + 2^5 + 2^6)$.
\inputminted[linenos,obeytabs,fontsize=\footnotesize]{oldasm}{../askhsh_2_i.8085}
\subsection*{Άσκηση 2(ii) α}
Χρονόμετρο και μετρητής διακοπών
\inputminted[linenos,obeytabs,fontsize=\footnotesize]{oldasm}{../askhsh_2_ii_a.8085}
\subsection*{Άσκηση 2(ii) β}
Χρονόμετρο και μετρητής bit της θύρας εισόδου
\inputminted[linenos,obeytabs,fontsize=\footnotesize]{oldasm}{../askhsh_2_ii_b.8085}

\end{document}
