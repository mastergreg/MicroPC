\documentclass[a4paper,10pt]{article} \usepackage{anysize}
\marginsize{2cm}{2cm}{1cm}{1cm}
%\textwidth 6.0in \textheight = 664pt
\usepackage{xltxtra}
\usepackage{xunicode} \usepackage{graphicx}
\usepackage{color} \usepackage{xgreek} \usepackage{fancyvrb}
\usepackage{colortbl}
\usepackage{minted}
\usepackage{listings}
\usepackage{enumitem} \usepackage{framed} \usepackage{relsize}
\usepackage{float} \setmainfont[Mapping=TeX-text]{CMU Serif}
\begin{document}

\begin{titlepage}
\begin{center}
\begin{figure}[h] 
     \includegraphics[width=0.2\textwidth]{title/ntua_logo}
\end{figure}
\vspace{1cm}
\begin{LARGE}\textbf{ΕΘΝΙΚΟ ΜΕΤΣΟΒΙΟ ΠΟΛΥΤΕΧΝΕΙΟ\\[1.5cm]}\end{LARGE}
\begin{Large}
ΣΧΟΛΗ ΗΜ\&ΜΥ\\
Εργαστήριο Μικροϋπολογιστών\\[2cm]
3\textsuperscript{η} Εργαστηριακή Άσκηση\\[0.5cm]
\LARGE{Γενικό Θέμα}\\[0.5cm]
\large{Ακ. έτος 2011-2012}\\
\end{Large}
\vfill
\begin{flushright}
\Large \textit{Ομάδα C07:}\\[1cm]
\begin{tabular}{l r}
{Ελένη \textsc{Ευαγγελάτου}}&
{Α.Μ.: 03108050}\\
{Γρηγόρης \textsc{Λύρας}}&
{Α.Μ.: 03109687}\\
{Βασιλεία \textsc{Φραγκιαδάκη}}&
{Α.Μ.: 03108026}\\
\end{tabular}
\end{flushright}

\large\today\\
\end{center}
\end{titlepage}




\begin{minipage}{0.4\textwidth}
\begin{flushleft}
\section*{Ζήτημα 7\textsuperscript{o}}
\end{flushleft}
\end{minipage}
\begin{minipage}{0.3\textwidth}
\begin{flushright}
\begin{minted}[fontsize=\footnotesize]{python}
>>> team = 7
>>> 1+(team-1) % 7
7
\end{minted}
\end{flushright}
\end{minipage}\\[0.5cm]

Για να έχουμε καταρχάς σωστές καταστάσεις, δηλαδή συνδυασμούς χρωμάτων των
leds, σχηματίζουμε τις εξής καταστάσεις που ζητά η εκφώνηση: 

\begin{table}[h]
\centering
	\begin{tabular}{|l|c|c|c|c|c|c|c|c|c|c|c|c|c|c|c|c|}
	\hline
	& 1 & 2 & 3 & 4 & 5 & 6 & 7 & 8 & 9 & 10 & 11 & 12 & 13 & 14 & 15\\
	\hline
	Κεντρικός & \multicolumn{10}{c|}{\cellcolor{green}Πράσινο} &
	\cellcolor{yellow}Κίτρινο & \multicolumn{4}{c|}{\cellcolor{red}Κόκκινο}\\
	\hline
	Πύλη & \multicolumn{11}{c}{\cellcolor{red}Κόκκινο} &
	\multicolumn{3}{c|}{\cellcolor{green}Πράσινο} & \cellcolor{yellow}Κίτρινο\\
	\hline
	\end{tabular}
	\caption{Διάγραμμα χρονισμού των φαναριών}
\end{table}

Βλέπουμε ποιά leds θέλουμε να ανάβουν, και αυτά που πρέπει να ανάβουν θα έχουν
λογικό 1 και όσα είναι σβηστά το λογικό 0. Επειδή βέβαια τα leds έχουν
αρνητική λογική, αυτά που θα ανάβουν θα έχουν το λογικό 0 και αυτά που θα
είναι σβηστά το λογικό 1. Έτσι για τον κάθε επιθυμητό συνδυασμό βάζουμε τον
αντίστοιχο δεκαεξαδικό αριθμό στην πύλη $3000_{16}$, για κάθε κατάσταση των
φαναριών που θέλουμε να απεικονίσουμε. Συγκεκριμένα:

\begin{table}[h]
\centering
	\begin{tabular}{|c|c|c|}
	\hline
	hex&Περιφεριακός & Πύλη ΕΜΠ\\
	\hline
	D7&\cellcolor{green}Πράσινο&\cellcolor{red}Κόκκινο\\
	\hline
	B7&\cellcolor{yellow}Κίτρινο&\cellcolor{red}Κόκκινο\\
	\hline
	7D&\cellcolor{red}Κόκκινο&\cellcolor{green}Πράσινο\\
	\hline
	7B&\cellcolor{red}Κόκκινο&\cellcolor{yellow}Κίτρινο\\
	\hline
	\end{tabular}
	\caption{Περιπτώσεις λειτουργίας φαναριών και δεκαεξαδικός κωδικός των
	σημάτων}
\end{table}

Όσον αφορά τον συγχρονισμό των καταστάσεων αυτών, θεωρούμε ένα κβάντο χρόνου
των $50 sec$. Και αυτό επειδή μας χρειάστηκε σε άλλο σημείο του κώδικα . Επεισή
όμως θέλουμε να έχουμε  $1 sec$ κβάντο χρόνου για την μέτρηση, φτιάξαμε μια
mini delay ,δηλαδή MDEL η οποία καθυστερεί $20* 50msec = 1 sec$. Με βάση αυτήν
την καθυστέρηση καλούμε την delb , με  καθυστέρηση $1 sec$ και την καλούμε τόσες
φορές όσα και τα δευτερόλεπτα που θέλουμε να διαρκεί ο εκάστοτε συνδυασμός
φαναριών.

Για το χρονόμετρο που μετράει αντίστροφα τον χρόνο, μετράμε στον καταχωρητή L
αντίστροφα ξεκινώντας από το $10$, έπειτα το $9$ και μετά αφαιρούμε $1$ σε κάθε
loop (LOOP1) κάνοντας μια καθυστέρηση $1 sec$ ώστε η μείωση να γίνεται όντως
κάθε $1 sec$ και έτσι τελικά να χρονομετρούμε. Το αποτέλεσμα το δείχνουμε στα 2
δεξιότερα 7segments με την βοήθεια της STDM. Για να μην δείχνουν την μέτρηση
του χρόνου στιγμιαία, καλούμε την DCD και αμέσως μετά την DELB με καθυστέρηση
$50 msec$ ώστε να κάνουμε ανανέωση (refresh) στο αποτέλεσμα που δείχνουν οι
7segments και να προλαβαίνει το ανθρώπινο μάτι να διαβάσει τον αριθμό που
απεικονίζεται.

Παρακάτω βρίσκεται ο κώδικάς μας.

\inputminted[linenos,obeytabs,frame=leftline,fontsize=\footnotesize]{oldasm}{files/fanaria.8085}

\end{document}
