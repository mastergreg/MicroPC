\documentclass[a4paper,10pt]{article} \usepackage{anysize}
\marginsize{2cm}{2cm}{1cm}{1cm}
%\textwidth 6.0in \textheight = 664pt
\usepackage{xltxtra}
\usepackage{xunicode} \usepackage{graphicx}
\usepackage{color} \usepackage{xgreek} \usepackage{fancyvrb}
\usepackage{minted}
\usepackage{listings}
\usepackage{enumitem} \usepackage{framed} \usepackage{relsize}
\usepackage{float} 
\setmainfont[Mapping=tex-text]{CMU Serif}
\begin{document}

\begin{titlepage}
\begin{center}
\begin{figure}[h] 
     \includegraphics[width=0.2\textwidth]{title/ntua_logo}
\end{figure}
\vspace{1cm}
\begin{LARGE}\textbf{ΕΘΝΙΚΟ ΜΕΤΣΟΒΙΟ ΠΟΛΥΤΕΧΝΕΙΟ\\[1.5cm]}\end{LARGE}
\begin{Large}
ΣΧΟΛΗ ΗΜ\&ΜΥ\\
Εργαστήριο Μικροϋπολογιστών\\[2cm]
3\textsuperscript{η} Εργαστηριακή Άσκηση\\[0.5cm]
\LARGE{Γενικό Θέμα}\\[0.5cm]
\large{Ακ. έτος 2011-2012}\\
\end{Large}
\vfill
\begin{flushright}
\Large \textit{Ομάδα C07:}\\[1cm]
\begin{tabular}{l r}
{Ελένη \textsc{Ευαγγελάτου}}&
{Α.Μ.: 03108050}\\
{Γρηγόρης \textsc{Λύρας}}&
{Α.Μ.: 03109687}\\
{Βασιλεία \textsc{Φραγκιαδάκη}}&
{Α.Μ.: 03108026}\\
\end{tabular}
\end{flushright}

\large\today\\
\end{center}
\end{titlepage}




\section*{} 
\subsection*{Άσκηση (i)}


\noindent Κυρίως κώδικας:
\inputminted[linenos,obeytabs,fontsize=\footnotesize]{c}{files/part1.S}
\subsection*{Άσκηση (ii)}

Σε αυτό το μέρος ο σκοπός μας είναι να αναβοσβήνουν τα leds ΡΑ0 έως PA7 αλλά
με συγκεκριμένη καθυστέρηση, την οποία δίνει ο χρήστης μέσω των dip switches B
του AVR. Για τον σκοπό αυτό με την ρουτίνα read\_delay\_A διαβάζουμε και
παίρνουμε τα 4 least significant bits και έπειτα προσθέτουμε 1 και και
πολλαπλασιάζουμε επί 100 κατά την δοθείσα σχέση $Delay = (1+x) * 100$. Για να
διαβάσουμε την καθυστέρηση για το σβήσιμο διαβάζουμε όπως πριν μόνο που
κάνουμε shift τέσσερις θέσεις αριστερά ώστε να απομονώσουμε τα επιθυμητά bits
και να εφαρμόσουμε την ίδια σχέση. Η κλήση για τις καθυστερήσεις είναι φυσικά
στο κυρίο πρόγραμμα, αμέσως μετά την κλήση για άναμμα ή σβήσιμο αντίστοιχα.
\noindent Κυρίως κώδικας:
\inputminted[linenos,obeytabs,fontsize=\footnotesize]{c}{files/part2.S}
\subsection*{Άσκηση (iii)}

Ζητείται να γράψουμε ένα πρόγραμμα σε C που να ανάβει το led0 που είναι
συνδεδεμένο στο bit0 της θύρας εξόδου PORTB το οποίο να μετακινείται κατάλληλα
ανάλογα με τα push buttons που πατάμε (SW0-4 της PORTD). Μεγαλύτερη
προτεραιότητα έχουν τα MSB buttons. Γι' αυτό το σκοπό περιμένουμε να γίνει 1
κάποιο button και στη συνέχεια μπαίνει σε loop μέχρι να γίνει μηδέν. Μετά
εκτελεί τη μετακίνηση που αντιστοιχεί σε αυτό. Εκτός αν πατηθεί ωστόσο button
υψηλότερης προτεραιότητας οπότε και πάμε στο αντίστοιχο loop που περιμένουμε
να γίνει 0. Για τις διάφορες μετακινήσεις κάναμε και τις απαραίτητες
διορθώσεις στην τιμή του led, ώστε να δείχνει την τιμή που περιμένουμε μετά
τις διάφορες ολισθήσεις.

\noindent Κυρίως κώδικας:
\inputminted[linenos,obeytabs,fontsize=\footnotesize]{c}{files/part3.c}

\end{document}
