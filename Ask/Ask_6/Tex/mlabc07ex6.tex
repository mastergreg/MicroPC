\documentclass[a4paper,10pt]{article} \usepackage{anysize}
\marginsize{2cm}{2cm}{1cm}{1cm}
%\textwidth 6.0in \textheight = 664pt
\usepackage{xltxtra}
\usepackage{xunicode}
\usepackage{graphicx}
\usepackage{color}
\usepackage{xgreek}
\usepackage{fancyvrb}
\usepackage{minted}
\usepackage{listings}
\usepackage{enumitem} 
\usepackage{framed} 
\usepackage{relsize}
\usepackage{float} 
\setmainfont[Mapping=tex-text]{CMU Serif}
\begin{document}

\def\thesubsection {Άσκηση (\roman{subsection})}

\begin{titlepage}
\begin{center}
\begin{figure}[h] 
     \includegraphics[width=0.2\textwidth]{title/ntua_logo}
\end{figure}
\vspace{1cm}
\begin{LARGE}\textbf{ΕΘΝΙΚΟ ΜΕΤΣΟΒΙΟ ΠΟΛΥΤΕΧΝΕΙΟ\\[1.5cm]}\end{LARGE}
\begin{Large}
ΣΧΟΛΗ ΗΜ\&ΜΥ\\
Εργαστήριο Μικροϋπολογιστών\\[2cm]
3\textsuperscript{η} Εργαστηριακή Άσκηση\\[0.5cm]
\LARGE{Γενικό Θέμα}\\[0.5cm]
\large{Ακ. έτος 2011-2012}\\
\end{Large}
\vfill
\begin{flushright}
\Large \textit{Ομάδα C07:}\\[1cm]
\begin{tabular}{l r}
{Ελένη \textsc{Ευαγγελάτου}}&
{Α.Μ.: 03108050}\\
{Γρηγόρης \textsc{Λύρας}}&
{Α.Μ.: 03109687}\\
{Βασιλεία \textsc{Φραγκιαδάκη}}&
{Α.Μ.: 03108026}\\
\end{tabular}
\end{flushright}

\large\today\\
\end{center}
\end{titlepage}




\section*{} 
\subsection{}

Στην άσκηση αυτή ζητείται στο χρονόμετρο που απεικονίζεται στην PORTA και
υλοποιείται με ταχύτητα 100 msec ανά μέτρηση, όταν ενεργοποιείται εξωτερική
διακοπή INT0, αν το PD0 είναι μηδέν, να απαριθμεί τις διακοπές και να
εμφανίζει το πλήθος τους στην PORTB. Γι' αυτό το σκοπό ενεργοποιούμε με τις
κατάλληλες εντολές τις διακοπές στο κυρίως πρόγραμμα και αρχικά βάζουμε το
.org 0x002 που επιτρέπει την εξυπηρέτηση της διακοπής INT0 στον Atmega16.

\noindent Κυρίως κώδικας:
\inputminted[linenos,obeytabs,fontsize=\footnotesize]{nasm}{files/part1.S}

\subsection{}
Σε αυτό το μέρος πραγματοποιήσαμε ένα πρόγραμμα μέτρησης το οποίο διακόπτεται
μόνο από την διακοπή INT1 (push button button PD3).  Το πρόγραμμα μέτρησης
εμφανίζεται στην σειρά led PA. Όταν έχουμε διακοπή πρέπει να διαβάζουμε από τα
dip switches B και να μετράμε πόσα είναι ON.  Το πόσα είναι ON φαίνεται στα
leds PC. Μετά την διακοπή συνεχίζεται κανονικά η μέτρηση στα leds PA.  Το
πρόγραμμα μέτρησης είναι ακριβώς όπως το δεθέν της εκφώνησης. Αυτή την φορά
απλά αναγνωρίζονται οι διακοπές INT1. Ο κώδικας εξυπηρέτησης της διακοπής
βρίσκεται στην label ISR1. Το πόσα dip switches είναι ON υπολογίζεται ως εξής:
διαβάζουμε από τον PINB, και κάνουμε shift δεξιά. Αν έχουμε carry, τότε
αυξάνουμε κατά 1 το άθροισμα των switches που είναι on. Διαφορετικά ελέγχουμε
το επόμενο ψηφίο. Αυτό γίνεται και για τα 8 bits.  Στο τέλος, για ασφάλεια
ώστε να έχουμε το αποτέλεσμα στα 4 least significant bits, κάνουμε μια bitwise
λογική πράξη and με τον αριθμό 00001111.

Έπειτα δείχνουμε το αποτέλεσμα στα leds PC. Ο κώδικάς μας: 

\noindent Κυρίως κώδικας:
\inputminted[linenos,obeytabs,fontsize=\footnotesize]{nasm}{files/part2.S}

\subsection{}
Σ' αυτή την άσκηση ζητείται να ανάβει το led PA1 για 3 sec με τη βοήθεια
χρονιστή, όταν ενεργοποιηθεί διακοπή INT0 ή είναι on το PA0.  Γι' αυτό το
σκοπό “σετάρουμε” το χρονιστή όταν ενεργοποιείται διακοπή ΙΝΤ0 ή το PΑ0 είναι
ένα και όταν προκληθεί υπερχείλιση ενεργοποιείται η αντίστοιχη διακοπή που
σβήνει το led. Αν ωστόσο ενεργοποιηθεί και άλλη διακοπή ή το PA0, σετάρεται
και πάλι ο χρονιστής στην αρχική του τιμή ώστε να ανανωθεί η καθυστέρηση. Να
σημειώσουμε ότι έχουμε βάλει τις απαράιτητες εντολές .org στην αρχή του
προγράμματος ώστε να είναι δυνατή η εξυπηρέτηση των διακοπών.

\noindent Κυρίως κώδικας:
\inputminted[linenos,obeytabs,fontsize=\footnotesize]{nasm}{files/part3.S}
\end{document}
