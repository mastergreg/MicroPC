\section{}

Σε αυτήν την άσκηση προσομοιώνουμε την λειτουργία ενός συστήματος συναγερμού.
Θέτουμε την θύρα Β ως είσοδο και έπειτα ελέγχουμε συνεχώς αν έχει γίνει
trigger στους αισθητήρες. Άπαξ και έχει γίνει, θέτουμε τον timer για να
αρχίσει να χρονομετρά και καλούμε την συνάρτηση getpass για να διαβάσουμε τον
κωδικό. Ελέγχουμε τα πλήκτρα που εισάγονται με παρόμοιο τρόπο όπως στην άσκηση
3, με την διαφορά ότι άμα έχουμε λάθος πλήκτρο τότε πηγαίνουμε στην ετικέτα
alarm\_on, όπου θέτουμε τον συναγερμό κατά τα ζητούμενα της άσκησης. Σε
περίπτωση που δοθεί ο σωστός κωδικός (τον οποίο έχουμε βάλει κατά σύμβαση
9807) απενεργοποιούμε τον συναγερμό (label alarm\_off), δηλαδή απενεργοποιούμε
τον timer και γραφουμε στην lcd οθόνη alarm off.

\noindent Κυρίως κώδικας:
\inputminted[linenos,obeytabs,fontsize=\footnotesize]{c}{files/part5.S}
