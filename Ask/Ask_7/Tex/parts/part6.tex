\section{}
Ζητούμενο της άσκησης αυτής είναι να διαβάσουμε ένα δεκαεξαδικό αριθμό από την
PORTA σε μορφή δυαδικού συμπληρώματος ως προς 2 και να γράψουμε την δεκαδική
του αναπαράσταση στην lcd οθόνη.  Η λογική που ακολουθούμε είναι η εξής:
διαβαζουμε το input και κάνουμε shift μια θέση δεξιά ώστε να δούμε αν θα
έχουμε 0 ή 1 για θετικό ή αρνητικό πρόσημο αντίστοιχα. Το πρόσημο αποθηκεύεται
στον καταχωρητή sign.  Έπειτα ανάλογα αν είναι ο αριθμός αρνητικός παίρνουμε
το συπμλήρωμά του ως προς 2, αλλιώς προχωρούμε στον υπολογισμό κατευθείαν.  Αν
είναι μεγαλύτερος του 100 τότε βάζουμε στον καταχωρητή ekat την μία
εκατοντάδα. Αλλιώς, και έπειτα, ελέγχουμε στο loop count\_dek το πόσες δεκάδες
έχουμε και αποθηκεύουμε το αποτέλεσμα στον καταχωρητή dek.  Αυτό που μας μένει
είναι οι μονάδες και το οποίο αποθηκευουμε στον καταχωρητή mon. Έπειτα
συνεχίζουμε στο να τα δείξουμε στην lcd οθόνη. Κάθε "κβάντο" πληροφορίας που
στέλνουμε αποθηκεύεται στον καταχωρητή r24 (τον οποίο έχουμε ονομάσει
quantum). Για την αποστολή βέβαια των εκατοντάδων και των δεκάδων βέβαια
φροντίζουμε να στέλνουμε τον ascii κωδικό των χαρακτήρων, προσθέτοντάς τους το
48 (dec).  Ακόμη έχουμε φροντίσει να ελέγχουμε αν είναι μηδενικές για να μην
εκτυπώνεται για παράδειγμα +009 αλλά +9. Πριν την αποστολή του αριθμού στην
οθόνη Έπειτα στέλνουμε διαδοχικά το sign,(αν υπάρχουν)τις εκατοντάδες, (αν
υπάρχουν) τις δεκάδες και τέλος τις μονάδες.

\noindent Κυρίως κώδικας:
\inputminted[linenos,obeytabs,fontsize=\footnotesize]{c}{files/part6.S}
